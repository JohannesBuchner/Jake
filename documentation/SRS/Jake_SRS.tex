\documentclass[english,a4paper,12pt]{report}
\usepackage[utf8]{inputenc}
\usepackage[ngerman]{babel}
\usepackage{multicol}
\usepackage[naturalnames]{hyperref}
\usepackage{verbatim}
%\usepackage[german,boxruled]{algorithm2e}
\usepackage{varioref}
\usepackage{amsmath, amsthm, amssymb}
\usepackage{graphicx}
\usepackage{fancyhdr}
\usepackage{lastpage}
%\usepackage[top=tlength, bottom=blength, left=llength, right=rlength]{geometry}
\usepackage{a4wide}

\newcommand{\tblheader}[1]{\textbf{#1}}

%\addtolength{\oddsidemargin}{-0.75in}
%\addtolength{\evensidemargin}{-0.75in}
%\addtolength{\marginparwidth}{-0.5in}
\setlength\parindent{0cm}

\setcounter{tocdepth}{2}

\begin{document}
\title{ Software Requirements Specification } 
\author{Simon Wallner}
\pagestyle{fancy}
\lhead{Jake}
\rhead{Software Requirements Secification}
\lfoot{Jake SRS}
%\cfoot{\today}
\cfoot{\number\year-\number\month-\number\day}
\rfoot{\thepage/\pageref{LastPage}}

\begin{titlepage}
\begin{center}
\LARGE{Project \textbf{Jake}}

\vfill
\Huge{\textbf{Software Requirements Specification}}
\end{center}

\vspace{2cm}
\textbf{Exec. Summary:} \\
SRS for \emph{Jake}

\vfill
%\vspace{2cm}
\begin{tabular}{ | l | l | }
\hline
 \tblheader{Autor:} & Simon Wallner \\
\hline
 \tblheader{Review:} & \\
\hline
 \tblheader{Gruppe:} & - \\
\hline
\end{tabular}

\vspace{2cm}

\begin{tabular}{ | l | l | p{3.5cm} | p{7cm} | }
\hline
  \tblheader{Nr} & \tblheader{Date} & \tblheader{Author} & \tblheader{Change} \\
\hline
1 & 2008-11-02 & Simon Wallner & document created \\
\hline

\end{tabular}
\thispagestyle{empty}
\end{titlepage}

\tableofcontents
\thispagestyle{fancy}
\clearpage

\pagestyle{fancy}

\section{Introduction}
\emph{Jake} is an application that simplifies sharing files in small project groups. It's main focus lies on sharing and replication of files over a network.

\subsection{Core Audience}
Jakes core audience are people with minimum to medium IT training, who are familiar with e-mail and internet application like browsers and instant messengers. 

\subsection{Usage Szenario}
A project group of 6-12 project members that shares about 5-100 files. All project members have internet access and are online, while they are working. Usually only one person works on a file at one time, therefore merging conflicts are rare. 

\section{Functional Requirements}
\subsection{Working with Projects }
Jake operates on the basis of \emph{projects}. A consists of \emph{project files} and \emph{project members}. All project files are rooted in the \emph{project folder}. This project folder exists locally in every project members file system. the project folder and its contents may be treated as any other file/folder in the file system. Changes in the file system are immediately reflected in the application.

It is possible to be connected to more than one project at a time within one instance of Jake. Every project MUST have a unique project folder. Files MUST only be associated with one project, therefore sharing a file with multiple projects is not supported.

Every project is identified by an unique ID and a non-unique \emph{project title}

\subsection{manage projects}
\subsection{offline content management}
\subsubsection{files}
\subsubsection{notes}
\subsubsection{metadata, tags}
\subsection{networking}
\subsection{synchronizing}
\subsubsection{web of trust}
\subsubsection{levels of trust}
\subsection{security}
\subsection{preferences}
\subsection{messaging}

\section{Nonfunctional Requirements}
\subsection{technical limitation and constraints}

\section{Further Documentation}
- use cases
- GUI specs/guidelines
- QA plan



\end{document}




