%    Beschreibung der Ziele die unser Projekt verfolgt. Dabei werden die einzelnen Ziele in Kategorien zusammengefasst, wie z.B. betriebswirtschaftliche Ziele, funktionale Ziele, Marketing-Ziele, soziale Ziele, Umfeldziele
%    Angabe der Lieferkomponenten, die komponenten die an den Kunden ausgeliefert werden
%    Angabe weiterer Komponenten die nicht an den Kunden ausgeliefert werden. technische Doku, interne Doku, …

\section{Arbeitsziele}

\subsection{Betriebswirtschaftliche Ziele}

Durch den Einsatz der Applikation wird es einfacher ad-hoc neue Dokumente der Projektgruppe zur Verfügung zu stellen oder Aktualisierungen an bestehenden zu Propagieren. Da Dieser Austausch nicht mehr per Mail geschieht wird die Übersicht über die Daten erhöht, und Versionskonflikte mit alten lokalen Versionen stark verringert.

Die Zeit die für das Organisieren und Durchsuche von alten Versionen eines Dokuments aufgewendet wurde kann nun für andere Tätigkeiten verwendet werden.

\subsection{Funktionale Ziele}

Durch den Einsatz der Applikation können Aktualisierungen an Dateien anderen Projektmitgliedern schneller zugänglich gemacht werden. Da die Projektmitglieder, sofern möglich, immer die aktuellsten Versionen zur Verfügung haben, ist eine dynamischere Arbeitsweise möglich, die stärker auf Zusammenarbeit setzt.

Die Benutzer werden bei der Lösung von Datei-Versionskonflikten von der Applikation unterstützt, wodurch diese einfacher zu handhaben sind.

Aktualisierungen können mit einem Klick an die Projektmitglieder propagiert werden, ohne dafür extra ein Mail zu schicken.

\subsection{Soziale Ziele}
Durch das Wegfallen der händischen Organisation alter Versionen und der händischen Verteilung von neuen wird der Arbeitsalltag der Projektmitglieder vereinfacht.

Durch den einfacheren Datenaustausch wird die kurzfristigere Zusammenarbeit der Projektmitglieder unterstützt.

