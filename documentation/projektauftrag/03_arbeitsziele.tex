%    Beschreibung der Ziele die unser Projekt verfolgt. Dabei werden die einzelnen Ziele in Kategorien zusammengefasst, wie z.B. betriebswirtschaftliche Ziele, funktionale Ziele, Marketing-Ziele, soziale Ziele, Umfeldziele
%    Angabe der Lieferkomponenten, die komponenten die an den Kunden ausgeliefert werden
%    Angabe weiterer Komponenten die nicht an den Kunden ausgeliefert werden. technische Doku, interne Doku, …

\section{Arbeitsziele}

\subsection{Betriebswirtschaftliche Ziele}
\begin{itemize}
\item Die Zeit die für das Verteilen, Speichern und Zusammenführen von verschiedenen Versionen eines Dokuments aufgewendet wurde kann nun für andere Tätigkeiten verwendet werden.
% Tags sind Metainformation. Meta-Tags gibts nicht
\item Durch Anhängen von Metainformation zu Datenobjekten wird die Übersichtlichkeit verbessert, was die Effizienz der Mitarbeiter eines Projektes erhöht.
\end{itemize}

\subsection{Funktionale Ziele}
\begin{itemize}
% Falls es noch keiner gesagt hat: Verben schreibt man klein, Substantiva groß. 
\item Durch den Einsatz der Applikation wird es einfacher ad-hoc neue Dokumente der Projektgruppe zur Verfügung zu stellen oder Aktualisierungen an Bestehenden zu propagieren. Da dieser Austausch nicht mehr per Mail geschieht wird die Übersicht über die Daten erhöht, und Versionskonflikte mit alten lokalen Versionen stark verringert.

\item Durch den Einsatz der Applikation können Aktualisierungen an Dateien anderen Projektmitgliedern schneller zugänglich gemacht werden. Da die Projektmitglieder, sofern möglich, immer die aktuellsten Versionen zur Verfügung haben, ist eine dynamischere Arbeitsweise möglich, die stärker auf Zusammenarbeit setzt.

% händisch = manuell
\item Da der Dateiaustausch nicht mehr per Mail geschieht müssen alte Versionen nicht mehr manuell organisiert werden, wodurch ein Versionschaos leichter vermieden werden kann.

\item Treten dennoch Datei-Versionskonflikte auf, wird der Benutzer von der Applikation bei deren Lösung unterstützt, wodurch diese einfacher zu handhaben sind und weniger Zeit in Anspruch nehmen.

\end{itemize}

\subsection{Soziale Ziele}
\begin{itemize}
\item Durch den einfacheren Datenaustausch wird die engere Zusammenarbeit der Projektmitglieder unterstützt.
\end{itemize}

\subsection{Lieferkomponenten}
Bei Projektabschluss werden folgende Komponenten übermittelt:
\begin{itemize}
\item die voll funktionale Applikation laut Anforderungsspezifikation als lauffähiges .jar Paket (benötigt JRE 1.6)
\item Benutzerhandbuch
\item Anforderungsspezifikation
\item Source Code
\item Technische Dokumentation
\end{itemize}

Die gesamte Dokumentation wird in einer Website zur Verfügung gestellt. Das Programm sowie die Dokumentation werden in englischer Sprache verfasst.

\subsection{Weitere Komponenten}
\begin{itemize}
\item Projektvorschlag
\item Projektauftrag
\item Projekt-Wiki
\item Dokumente der internen Projektorganisation
\item Artefakte des laufenden Projektmanagement
\item Stundenlisten
\item Projekttagebuch
\item Protokolle
\end{itemize}
