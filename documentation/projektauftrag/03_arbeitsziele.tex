%    Beschreibung der Ziele die unser Projekt verfolgt. Dabei werden die einzelnen Ziele in Kategorien zusammengefasst, wie z.B. betriebswirtschaftliche Ziele, funktionale Ziele, Marketing-Ziele, soziale Ziele, Umfeldziele
%    Angabe der Lieferkomponenten, die komponenten die an den Kunden ausgeliefert werden
%    Angabe weiterer Komponenten die nicht an den Kunden ausgeliefert werden. technische Doku, interne Doku, …

\section{Arbeitsziele}

\subsection{Betriebswirtschaftliche Ziele}
% Geld und Zeit-Einsparungen
\begin{itemize}
\item Die Zeit die für das Organisieren und Durchsuche von alten Versionen eines Dokuments aufgewendet wurde kann nun für andere Tätigkeiten verwendet werden.
\end{itemize}

\subsection{Funktionale Ziele}
% Keine Features!
% Was haben die Leute davon, was sie jetzt nicht haben (ohne techn. Details)
\begin{itemize}
\item Durch den Einsatz der Applikation wird es einfacher ad-hoc neue Dokumente der Projektgruppe zur Verfügung zu stellen oder Aktualisierungen an bestehenden zu Propagieren. Da die Projektmitglieder, sofern möglich, immer die aktuellsten Versionen zur Verfügung haben, ist eine dynamischere Arbeitsweise möglich, die stärker auf Zusammenarbeit setzt.

\item Durch das Wegfallen der händischen Organisation verschiedener Dateiversionen wird der Arbeitsalltag der Projektmitglieder vereinfacht. Da der Dateiaustausch nicht mehr per Mail geschieht wird die Übersichtlichkeit erhöht, und Versionskonflikte mit alten lokalen Versionen bzw. das Wiederkehren alter Fehler stark verringert.
\end{itemize}

\subsection{Soziale Ziele}
% Wir haben keine wirklichen Sozialen Ziele, aber da in der Artefaktenbeschreibung
\begin{itemize}
\item Durch den einfacheren Datenaustausch wird die kurzfristigere Zusammenarbeit der Projektmitglieder unterstützt.
\end{itemize}

\subsection{Lieferkomponenten}
Die Software wird als ausführbare jar-Datei geliefert. Diese ist mit einer aktuellen Java-Runtime Machine (ab 1.6) lauffähig.
\subsection{Weitere Komponenten}
 - Projektdokumentation
