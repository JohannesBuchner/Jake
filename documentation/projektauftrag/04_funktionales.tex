% funktionale Anforderungen, Anwendungsfälle
%    Eine Liste von funktionalen Anforderungen oder eine Liste von User-Level Use Cases. Dabei sind nicht die ausführlichen Use Cases selbst gemeint (die kommen erst später) sondern lediglich gut beschreibende one-liner, eine Weiterführung der Featureliste. z.B. Datei zum Projektpool hinzufügen, neues Projekt erstellen, neues Projektmitglied hinzufügen

\section{Funktionale Anforderungen}
* Authentifizierung bei Programmstart mittels User/Passwort
* Erstellen eines Projektes (Dateipool)
* Hinzuf\"ugen von Benutzern zu einem Projektpool ( mittels Netzwerkservice)
* Teilnahme an einem Projekt best\"atigen/ablehnen
* Spezifikation eines Ordners auf der lokalen Festplatte f\"ur den Dateipool
* Hinzuf\"ugen von Dateien zu einem Dateipool
* Erstellen/Verwalten von Labels/Tags
* Zuweisen/Entfernen von Labels zu einer/mehreren Dateien
* Suchen von Dateien mittels Name
* Suchen von Dateien mittels Label/Tag
* Erstellen von Notizen
* Hinzuf\"ugen von Metadaten zu Dateien/Notizen
* Verbinden zum Projekt (sofern zuerst offline gearbeitet wurde)
* Verbindung zum Projekt trennen (sofern zuerst online gearbeitet wurde)
* Explizites Synchronisieren des Projektes (\"Anderungen herunterladen / ver\"offentlichen)

