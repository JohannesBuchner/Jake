% funktionale Anforderungen, Anwendungsfälle
%    Eine Liste von funktionalen Anforderungen oder eine Liste von User-Level Use Cases. Dabei sind nicht die ausführlichen Use Cases selbst gemeint (die kommen erst später) sondern lediglich gut beschreibende one-liner, eine Weiterführung der Featureliste. z.B. Datei zum Projektpool hinzufügen, neues Projekt erstellen, neues Projektmitglied hinzufügen

\section{Use Cases}
% Es soll entweder eine Liste von Anforderungen oder eine Liste von (User-Level)
% Anwendungsfällen erstellt werden. Dabei soll die Featureliste aus dem
% Projektvorschlag hilfreich sein. Egal ob Anwendungsfälle oder Anforderungen, beide
% können Gewichtet und aufgeteilt werden um eine Entscheidungsgrundlage für das
% Arbeitsprogramm und die WBS zu erstellen. (z.B. Need-to-Have, Nice-to-Have)
% BEISPIEL
% Anforderungen: User Interface soll Eingabemasken zur Verwaltung von Studenten
% und Prüfungen implementieren. Studentenlisten sollen über einen grafischen
% Dateiauswahl- bzw. Dateispeicherdialog als XML geöffnet und gespeichert werden
% können.
% Anwendungsfälle: Studenten verwaten/exportieren, Prüfung anmelden/absolvieren

% Hier bitte ENTWEDER alles Anwendungsfälle ODER alles Features schreiben
% nicht mischen!

\begin{itemize}

\item Projekte verwalten
\item Projektmitglieder verwalten
\item Dateien/Ordner zum Projektdatenpool hinzufügen
\item Dateien/Ordner aus dem Projektdatenpool entfernen
\item Datei aus dem Projektdatenpool mit einer externen Applikation öffnen
\item Notizen organisieren
\item Metadaten für das Projekt organisieren
\item Metadaten für Daten organisieren
\item Metadaten für Projektmitglieder organisieren
\item lokale Änderungen an Projektmitglieder propagieren
\item aktualisierte Versionen von Projektmitgliedern holen
\item Versionskonflikt lösen
\item Nachrichten an Projektmitglieder schicken
\item Nachrichten empfangen

\end{itemize}