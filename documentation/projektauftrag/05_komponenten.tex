% Komponentendiagramm des Projektes
\section{Komponentendiagramm}

%% img component_diagram.png

Das Projekt ist in vorwiegend 6 Komponenten aufgeteilt. 
\begin{itemize}
	\item Graphical User Interface (GUI)
	\item Core
	\item Database Persistence (DAO)
	\item Synchronisation Services
	\item File System Services
	\item Network Service
\end{itemize}

\subsection{Graphical User Interface}
Das Graphical User Interface (GUI) ist die grafische Benutzeroberfläche, mit welcher der Endanwender arbeitet.
Die GUI ermöglicht Zugriff auf alle von der ``Core"-Komponente für Endbenutzer zur Verfügung gestellten
Funktionalitäten.

\subsection{Core}
In der Core-Komponente befindet sich der Großteil der Logik der Applikation. Hier wird entschieden, was bei
bestimmten User-Eingaben (z.B. das Drücken des Buttons ``Synchronisieren") gemacht werden soll. Die
Core Komponente deligiert die vom User gewünschte Aktion an die einzelnen Komponenten, fügt die Ergebnisse
der Aktionen zusammen und gibt diese zurück an die GUI. Der Core reagiert aber auch auf Events, die durch
die einzelnen Komponenten gemeldet werden, y.B. wenn eine Datei im Dateisystem geändert wird, wird dies über
den Core an die GUI gemeldet und je nach User Interaktion weiter an die Synchronisationskomponente geleitet.

\subsection{Database Persistence}
Die Database Persistence Komponente abstrahiert den Zugriff auf die Datenbank und ermöglicht es, definierte
Aktionen einfach und geregelt (z.B. mittels Transaktionen) auszuführen. Es wird das Konzept des Data Hiding
umgesetzt, wodurch erreicht werden kann, dass der Core bzw. im schlimmsten Falle andere Clients keinen direkten
Zugriff auf die Datenbank erhalten, sondern nur auf die spezifizierten und notwendigen Funktionen. 

\subsection{Synchronisation Services}
Die Synchronisationskomponente arbeitet sehr eng mit dem Core zusammen. Der Core übermittelt beispielsweise
eine Änderung einer Datei, welche durch den User bzw. automatisch (je nach Einstellung) bestätigt wurde.
Die Synchronisationskomponente beinhaltet nun verschiedene Strategien, wie diese Änderungen an andere
Projektmitglieder/Clients propagiert werden. 

\subsection{File System Services}
Die File System Services sind die Schnittstelle des Programmes zum Dateisystem des Benutzers. Dieser
Service kapselt den kompleten Zugriff auf Dateien im Dateisystem, sodass diese vom Programm wie jegliche
andere Objekte (z.B. Notizen) verwendet werden können. Außerdem kann der Dateisystem Service durch
entsprechende Strategien feststellen, ob Dateien geändert wurden oder in die Projektordnerstruktur
kopiert wurden und dies dem Core mitteilen, welcher wiederum entsprechende Aktionen veranlasst.

\subsection{Network Service}
Der Network Service behandelt alle Netzwerktätigkeiten des Programmes. Der Service kapselt beispielsweise
vollständig die Kommunikation der Clients auf Netzwerkebene und gibt die entsprechenden Nachrichten an
den Core weiter. So ist es leicht möglich, verschiedene Netzwerkbackends (z.B. XMPP oder RMI) zu unterstützen,
welche für den Core und somit für den Benutzer transparent sind. Außerdem wird die Authentifizierung der
Nutzer in dieser Komponente durchgeführt.

